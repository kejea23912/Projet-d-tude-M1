% Options for packages loaded elsewhere
\PassOptionsToPackage{unicode}{hyperref}
\PassOptionsToPackage{hyphens}{url}
\documentclass[
  ignorenonframetext,
]{beamer}
\newif\ifbibliography
\usepackage{pgfpages}
\setbeamertemplate{caption}[numbered]
\setbeamertemplate{caption label separator}{: }
\setbeamercolor{caption name}{fg=normal text.fg}
\beamertemplatenavigationsymbolsempty
% remove section numbering
\setbeamertemplate{part page}{
  \centering
  \begin{beamercolorbox}[sep=16pt,center]{part title}
    \usebeamerfont{part title}\insertpart\par
  \end{beamercolorbox}
}
\setbeamertemplate{section page}{
  \centering
  \begin{beamercolorbox}[sep=12pt,center]{section title}
    \usebeamerfont{section title}\insertsection\par
  \end{beamercolorbox}
}
\setbeamertemplate{subsection page}{
  \centering
  \begin{beamercolorbox}[sep=8pt,center]{subsection title}
    \usebeamerfont{subsection title}\insertsubsection\par
  \end{beamercolorbox}
}
% Prevent slide breaks in the middle of a paragraph
\widowpenalties 1 10000
\raggedbottom
\AtBeginPart{
  \frame{\partpage}
}
\AtBeginSection{
  \ifbibliography
  \else
    \frame{\sectionpage}
  \fi
}
\AtBeginSubsection{
  \frame{\subsectionpage}
}
\usepackage{iftex}
\ifPDFTeX
  \usepackage[T1]{fontenc}
  \usepackage[utf8]{inputenc}
  \usepackage{textcomp} % provide euro and other symbols
\else % if luatex or xetex
  \usepackage{unicode-math} % this also loads fontspec
  \defaultfontfeatures{Scale=MatchLowercase}
  \defaultfontfeatures[\rmfamily]{Ligatures=TeX,Scale=1}
\fi
\usepackage{lmodern}
\usetheme[]{Madrid}
\usecolortheme[]{dolphin}
\ifPDFTeX\else
  % xetex/luatex font selection
\fi
% Use upquote if available, for straight quotes in verbatim environments
\IfFileExists{upquote.sty}{\usepackage{upquote}}{}
\IfFileExists{microtype.sty}{% use microtype if available
  \usepackage[]{microtype}
  \UseMicrotypeSet[protrusion]{basicmath} % disable protrusion for tt fonts
}{}
\makeatletter
\@ifundefined{KOMAClassName}{% if non-KOMA class
  \IfFileExists{parskip.sty}{%
    \usepackage{parskip}
  }{% else
    \setlength{\parindent}{0pt}
    \setlength{\parskip}{6pt plus 2pt minus 1pt}}
}{% if KOMA class
  \KOMAoptions{parskip=half}}
\makeatother
\usepackage{color}
\usepackage{fancyvrb}
\newcommand{\VerbBar}{|}
\newcommand{\VERB}{\Verb[commandchars=\\\{\}]}
\DefineVerbatimEnvironment{Highlighting}{Verbatim}{commandchars=\\\{\}}
% Add ',fontsize=\small' for more characters per line
\usepackage{framed}
\definecolor{shadecolor}{RGB}{248,248,248}
\newenvironment{Shaded}{\begin{snugshade}}{\end{snugshade}}
\newcommand{\AlertTok}[1]{\textcolor[rgb]{0.94,0.16,0.16}{#1}}
\newcommand{\AnnotationTok}[1]{\textcolor[rgb]{0.56,0.35,0.01}{\textbf{\textit{#1}}}}
\newcommand{\AttributeTok}[1]{\textcolor[rgb]{0.13,0.29,0.53}{#1}}
\newcommand{\BaseNTok}[1]{\textcolor[rgb]{0.00,0.00,0.81}{#1}}
\newcommand{\BuiltInTok}[1]{#1}
\newcommand{\CharTok}[1]{\textcolor[rgb]{0.31,0.60,0.02}{#1}}
\newcommand{\CommentTok}[1]{\textcolor[rgb]{0.56,0.35,0.01}{\textit{#1}}}
\newcommand{\CommentVarTok}[1]{\textcolor[rgb]{0.56,0.35,0.01}{\textbf{\textit{#1}}}}
\newcommand{\ConstantTok}[1]{\textcolor[rgb]{0.56,0.35,0.01}{#1}}
\newcommand{\ControlFlowTok}[1]{\textcolor[rgb]{0.13,0.29,0.53}{\textbf{#1}}}
\newcommand{\DataTypeTok}[1]{\textcolor[rgb]{0.13,0.29,0.53}{#1}}
\newcommand{\DecValTok}[1]{\textcolor[rgb]{0.00,0.00,0.81}{#1}}
\newcommand{\DocumentationTok}[1]{\textcolor[rgb]{0.56,0.35,0.01}{\textbf{\textit{#1}}}}
\newcommand{\ErrorTok}[1]{\textcolor[rgb]{0.64,0.00,0.00}{\textbf{#1}}}
\newcommand{\ExtensionTok}[1]{#1}
\newcommand{\FloatTok}[1]{\textcolor[rgb]{0.00,0.00,0.81}{#1}}
\newcommand{\FunctionTok}[1]{\textcolor[rgb]{0.13,0.29,0.53}{\textbf{#1}}}
\newcommand{\ImportTok}[1]{#1}
\newcommand{\InformationTok}[1]{\textcolor[rgb]{0.56,0.35,0.01}{\textbf{\textit{#1}}}}
\newcommand{\KeywordTok}[1]{\textcolor[rgb]{0.13,0.29,0.53}{\textbf{#1}}}
\newcommand{\NormalTok}[1]{#1}
\newcommand{\OperatorTok}[1]{\textcolor[rgb]{0.81,0.36,0.00}{\textbf{#1}}}
\newcommand{\OtherTok}[1]{\textcolor[rgb]{0.56,0.35,0.01}{#1}}
\newcommand{\PreprocessorTok}[1]{\textcolor[rgb]{0.56,0.35,0.01}{\textit{#1}}}
\newcommand{\RegionMarkerTok}[1]{#1}
\newcommand{\SpecialCharTok}[1]{\textcolor[rgb]{0.81,0.36,0.00}{\textbf{#1}}}
\newcommand{\SpecialStringTok}[1]{\textcolor[rgb]{0.31,0.60,0.02}{#1}}
\newcommand{\StringTok}[1]{\textcolor[rgb]{0.31,0.60,0.02}{#1}}
\newcommand{\VariableTok}[1]{\textcolor[rgb]{0.00,0.00,0.00}{#1}}
\newcommand{\VerbatimStringTok}[1]{\textcolor[rgb]{0.31,0.60,0.02}{#1}}
\newcommand{\WarningTok}[1]{\textcolor[rgb]{0.56,0.35,0.01}{\textbf{\textit{#1}}}}
\usepackage{graphicx}
\makeatletter
\newsavebox\pandoc@box
\newcommand*\pandocbounded[1]{% scales image to fit in text height/width
  \sbox\pandoc@box{#1}%
  \Gscale@div\@tempa{\textheight}{\dimexpr\ht\pandoc@box+\dp\pandoc@box\relax}%
  \Gscale@div\@tempb{\linewidth}{\wd\pandoc@box}%
  \ifdim\@tempb\p@<\@tempa\p@\let\@tempa\@tempb\fi% select the smaller of both
  \ifdim\@tempa\p@<\p@\scalebox{\@tempa}{\usebox\pandoc@box}%
  \else\usebox{\pandoc@box}%
  \fi%
}
% Set default figure placement to htbp
\def\fps@figure{htbp}
\makeatother
\setlength{\emergencystretch}{3em} % prevent overfull lines
\providecommand{\tightlist}{%
  \setlength{\itemsep}{0pt}\setlength{\parskip}{0pt}}
\usepackage{bookmark}
\IfFileExists{xurl.sty}{\usepackage{xurl}}{} % add URL line breaks if available
\urlstyle{same}
\hypersetup{
  pdftitle={exemple de presentation},
  pdfauthor={Kenny Jean-elie; Ibrahima Caba Bah; Fatou Diop Ndeye},
  hidelinks,
  pdfcreator={LaTeX via pandoc}}

\title{exemple de presentation}
\author{Kenny Jean-elie \and Ibrahima Caba Bah \and Fatou Diop Ndeye}
\date{2025-11-22}

\begin{document}
\frame{\titlepage}

\section{Introduction}\label{introduction}

\begin{frame}{Introduction}
Bienvenue dans cette mini présentation Beamer avec R Markdown.
\end{frame}

\begin{frame}{Objectif}
\phantomsection\label{objectif}
\begin{itemize}
\tightlist
\item
  Montrer une slide avec deux colonnes
\item
  Combiner un graphique et un commentaire
\end{itemize}
\end{frame}

\begin{frame}{Slide avec deux colonnes}
\phantomsection\label{slide-avec-deux-colonnes}
\begin{columns}[T]
\begin{column}{0.5\linewidth}
\pandocbounded{\includegraphics[keepaspectratio]{Presentation_files/figure-beamer/unnamed-chunk-1-1.pdf}}
\end{column}

\begin{column}{0.5\linewidth}
``Cette diapositive utilise deux colonnes. À gauche, nous avons un
graphique représentant la relation entre la température et la pression.
À droite, nous avons ce commentaire explicatif.''
\end{column}
\end{columns}
\end{frame}

\section{Importation des librairies}\label{importation-des-librairies}

\begin{frame}[fragile]{Importation des librairies}
\begin{Shaded}
\begin{Highlighting}[]
\FunctionTok{library}\NormalTok{(tidyverse)}
\end{Highlighting}
\end{Shaded}

\begin{verbatim}
## -- Attaching core tidyverse packages ------------------------ tidyverse 2.0.0 --
## v dplyr     1.1.4     v readr     2.1.6
## v forcats   1.0.1     v stringr   1.6.0
## v ggplot2   4.0.1     v tibble    3.3.0
## v lubridate 1.9.4     v tidyr     1.3.1
## v purrr     1.2.0     
## -- Conflicts ------------------------------------------ tidyverse_conflicts() --
## x dplyr::filter() masks stats::filter()
## x dplyr::lag()    masks stats::lag()
## i Use the conflicted package (<http://conflicted.r-lib.org/>) to force all conflicts to become errors
\end{verbatim}
\end{frame}

\section{Importaion de la base de
données}\label{importaion-de-la-base-de-donnuxe9es}

\begin{frame}[fragile]{Importaion de la base de données}
\begin{Shaded}
\begin{Highlighting}[]
\NormalTok{df}\OtherTok{\textless{}{-}}\FunctionTok{read.csv}\NormalTok{(}\StringTok{"basse\_final.csv"}\NormalTok{,}\AttributeTok{sep =} \StringTok{","}\NormalTok{, }\AttributeTok{dec =} \StringTok{"."}\NormalTok{,}\AttributeTok{header =} \ConstantTok{TRUE}\NormalTok{)[,}\SpecialCharTok{{-}}\DecValTok{1}\NormalTok{]}
\FunctionTok{head}\NormalTok{(df)}
\end{Highlighting}
\end{Shaded}

\begin{verbatim}
##   Consommation.annuelle.totale.de.l.adresse..MWh.
## 1                                          25.316
## 2                                          17.846
## 3                                         138.250
## 4                                         221.062
## 5                                         364.091
## 6                                          56.757
##   Consommation.annuelle.moyenne.par.logement.de.l.adresse..MWh.
## 1                                                         1.808
## 2                                                         1.622
## 3                                                         2.194
## 4                                                         3.509
## 5                                                         1.291
## 6                                                         2.838
##   Consommation.annuelle.moyenne.de.la.commune..MWh.
## 1                                             6.317
## 2                                             6.317
## 3                                             2.631
## 4                                             2.631
## 5                                             2.631
## 6                                             2.631
##                                 adresse    numero_dpe etiquette_dpe
## 1 175 RUE DE LA PLAINE 383 RUY MONTCEAU 2438E4219851O             C
## 2 159 RUE DE LA PLAINE 383 RUY MONTCEAU 2438E4219636H             C
## 3     17 SQUARE SAINT CHARLES 750 PARIS 2475E1185567T             F
## 4          7 RUE SAINT CLAUDE 750 PARIS 2475E2805868A             E
## 5        39 RUE SAINT FARGEAU 750 PARIS 2475E2267676I             D
## 6      48 RUE SAINT FERDINAND 750 PARIS 2375E4566733O             D
##   etiquette_ges annee_construction surface_habitable_logement conso_5.usages_ef
## 1             C               2021                       49.1            3405.2
## 2             C               2021                       81.8            6671.5
## 3             B               1980                       32.0            4944.4
## 4             B               1945                       30.2            4094.2
## 5             D               1974                       27.1            5144.8
## 6             B               1945                       22.7            2283.1
##   emission_ges_5_usages cout_total_5_usages nombre_appartement
## 1                 713.1                 429                 11
## 2                1431.4                 782                 11
## 3                 360.0                 892                 65
## 4                 301.9                 934                 71
## 5                 915.1                 494                278
## 6                 159.0                 591                 12
\end{verbatim}
\end{frame}

\section{ACP}\label{acp}

\begin{frame}{inertie}
\phantomsection\label{inertie}
\end{frame}

\section{AFC}\label{afc}

\begin{frame}{relation\ldots.}
\phantomsection\label{relation.}
blannn
\end{frame}

\section{ACM}\label{acm}

\begin{frame}[fragile]{ACM}
\begin{Shaded}
\begin{Highlighting}[]
\FunctionTok{str}\NormalTok{(df)}
\end{Highlighting}
\end{Shaded}

\begin{verbatim}
## 'data.frame':    7935 obs. of  13 variables:
##  $ Consommation.annuelle.totale.de.l.adresse..MWh.              : num  25.3 17.8 138.2 221.1 364.1 ...
##  $ Consommation.annuelle.moyenne.par.logement.de.l.adresse..MWh.: num  1.81 1.62 2.19 3.51 1.29 ...
##  $ Consommation.annuelle.moyenne.de.la.commune..MWh.            : num  6.32 6.32 2.63 2.63 2.63 ...
##  $ adresse                                                      : chr  "175 RUE DE LA PLAINE 383 RUY MONTCEAU" "159 RUE DE LA PLAINE 383 RUY MONTCEAU" "17 SQUARE SAINT CHARLES 750 PARIS" "7 RUE SAINT CLAUDE 750 PARIS" ...
##  $ numero_dpe                                                   : chr  "2438E4219851O" "2438E4219636H" "2475E1185567T" "2475E2805868A" ...
##  $ etiquette_dpe                                                : chr  "C" "C" "F" "E" ...
##  $ etiquette_ges                                                : chr  "C" "C" "B" "B" ...
##  $ annee_construction                                           : int  2021 2021 1980 1945 1974 1945 1910 1967 1947 1973 ...
##  $ surface_habitable_logement                                   : num  49.1 81.8 32 30.2 27.1 22.7 142 50 83.3 75.1 ...
##  $ conso_5.usages_ef                                            : num  3405 6672 4944 4094 5145 ...
##  $ emission_ges_5_usages                                        : num  713 1431 360 302 915 ...
##  $ cout_total_5_usages                                          : num  429 782 892 934 494 ...
##  $ nombre_appartement                                           : int  11 11 65 71 278 12 33 21 12 83 ...
\end{verbatim}
\end{frame}

\begin{frame}[fragile]{Conversion des variables qualitatives en factor}
\phantomsection\label{conversion-des-variables-qualitatives-en-factor}
\begin{Shaded}
\begin{Highlighting}[]
\NormalTok{qualitative\_cols}\OtherTok{\textless{}{-}}\FunctionTok{c}\NormalTok{(}\StringTok{"adresse"}\NormalTok{,}\StringTok{"numero\_dpe"}\NormalTok{,}\StringTok{"etiquette\_dpe"}\NormalTok{,}\StringTok{"etiquette\_ges"}\NormalTok{)}
\NormalTok{df[qualitative\_cols]}\OtherTok{\textless{}{-}}\FunctionTok{lapply}\NormalTok{(df[qualitative\_cols], as.factor)}
\end{Highlighting}
\end{Shaded}
\end{frame}

\begin{frame}{}
\phantomsection\label{section}
\end{frame}

\section{}\label{section-1}

\begin{frame}{blabla}
\phantomsection\label{blabla}
\end{frame}

\section{clasification}\label{clasification}

\begin{frame}{reste}
\phantomsection\label{reste}
\end{frame}

\end{document}
