% Options for packages loaded elsewhere
\PassOptionsToPackage{unicode}{hyperref}
\PassOptionsToPackage{hyphens}{url}
\documentclass[
  ignorenonframetext,
]{beamer}
\newif\ifbibliography
\usepackage{pgfpages}
\setbeamertemplate{caption}[numbered]
\setbeamertemplate{caption label separator}{: }
\setbeamercolor{caption name}{fg=normal text.fg}
\beamertemplatenavigationsymbolsempty
% remove section numbering
\setbeamertemplate{part page}{
  \centering
  \begin{beamercolorbox}[sep=16pt,center]{part title}
    \usebeamerfont{part title}\insertpart\par
  \end{beamercolorbox}
}
\setbeamertemplate{section page}{
  \centering
  \begin{beamercolorbox}[sep=12pt,center]{section title}
    \usebeamerfont{section title}\insertsection\par
  \end{beamercolorbox}
}
\setbeamertemplate{subsection page}{
  \centering
  \begin{beamercolorbox}[sep=8pt,center]{subsection title}
    \usebeamerfont{subsection title}\insertsubsection\par
  \end{beamercolorbox}
}
% Prevent slide breaks in the middle of a paragraph
\widowpenalties 1 10000
\raggedbottom
\AtBeginPart{
  \frame{\partpage}
}
\AtBeginSection{
  \ifbibliography
  \else
    \frame{\sectionpage}
  \fi
}
\AtBeginSubsection{
  \frame{\subsectionpage}
}
\usepackage{iftex}
\ifPDFTeX
  \usepackage[T1]{fontenc}
  \usepackage[utf8]{inputenc}
  \usepackage{textcomp} % provide euro and other symbols
\else % if luatex or xetex
  \usepackage{unicode-math} % this also loads fontspec
  \defaultfontfeatures{Scale=MatchLowercase}
  \defaultfontfeatures[\rmfamily]{Ligatures=TeX,Scale=1}
\fi
\usepackage{lmodern}
\usetheme[]{Madrid}
\usecolortheme[]{dolphin}
\ifPDFTeX\else
  % xetex/luatex font selection
\fi
% Use upquote if available, for straight quotes in verbatim environments
\IfFileExists{upquote.sty}{\usepackage{upquote}}{}
\IfFileExists{microtype.sty}{% use microtype if available
  \usepackage[]{microtype}
  \UseMicrotypeSet[protrusion]{basicmath} % disable protrusion for tt fonts
}{}
\makeatletter
\@ifundefined{KOMAClassName}{% if non-KOMA class
  \IfFileExists{parskip.sty}{%
    \usepackage{parskip}
  }{% else
    \setlength{\parindent}{0pt}
    \setlength{\parskip}{6pt plus 2pt minus 1pt}}
}{% if KOMA class
  \KOMAoptions{parskip=half}}
\makeatother
\usepackage{graphicx}
\makeatletter
\newsavebox\pandoc@box
\newcommand*\pandocbounded[1]{% scales image to fit in text height/width
  \sbox\pandoc@box{#1}%
  \Gscale@div\@tempa{\textheight}{\dimexpr\ht\pandoc@box+\dp\pandoc@box\relax}%
  \Gscale@div\@tempb{\linewidth}{\wd\pandoc@box}%
  \ifdim\@tempb\p@<\@tempa\p@\let\@tempa\@tempb\fi% select the smaller of both
  \ifdim\@tempa\p@<\p@\scalebox{\@tempa}{\usebox\pandoc@box}%
  \else\usebox{\pandoc@box}%
  \fi%
}
% Set default figure placement to htbp
\def\fps@figure{htbp}
\makeatother
\setlength{\emergencystretch}{3em} % prevent overfull lines
\providecommand{\tightlist}{%
  \setlength{\itemsep}{0pt}\setlength{\parskip}{0pt}}
\usepackage{booktabs}
\usepackage{longtable}
\usepackage{array}
\usepackage{multirow}
\usepackage{wrapfig}
\usepackage{float}
\usepackage{colortbl}
\usepackage{pdflscape}
\usepackage{tabu}
\usepackage{threeparttable}
\usepackage{threeparttablex}
\usepackage[normalem]{ulem}
\usepackage{makecell}
\usepackage{xcolor}
\usepackage{bookmark}
\IfFileExists{xurl.sty}{\usepackage{xurl}}{} % add URL line breaks if available
\urlstyle{same}
\hypersetup{
  pdftitle={exemple de presentation},
  pdfauthor={Kenny Jean-elie; Ibrahima Caba Bah; Fatou Diop Ndeye},
  hidelinks,
  pdfcreator={LaTeX via pandoc}}

\title{exemple de presentation}
\author{Kenny Jean-elie \and Ibrahima Caba Bah \and Fatou Diop Ndeye}
\date{2025-12-11}

\begin{document}
\frame{\titlepage}

\section{Introduction}\label{introduction}

\begin{frame}{Introduction}
Bienvenue dans cette mini présentation Beamer avec R Markdown.
\end{frame}

\begin{frame}{Objectif}
\phantomsection\label{objectif}
\begin{itemize}
\tightlist
\item
  Montrer une slide avec deux colonnes
\item
  Combiner un graphique et un commentaire
\end{itemize}
\end{frame}

\begin{frame}{Slide avec deux colonnes}
\phantomsection\label{slide-avec-deux-colonnes}
\begin{columns}[T]
\begin{column}{0.5\linewidth}
\pandocbounded{\includegraphics[keepaspectratio]{Presentation_files/figure-beamer/unnamed-chunk-1-1.pdf}}
\end{column}

\begin{column}{0.5\linewidth}
``Cette diapositive utilise deux colonnes. À gauche, nous avons un
graphique représentant la relation entre la température et la pression.
À droite, nous avons ce commentaire explicatif.''
\end{column}
\end{columns}
\end{frame}

\section{ACP}\label{acp}

\begin{frame}[fragile]{inertie}
\phantomsection\label{inertie}
\begin{verbatim}
## [1] 7156
\end{verbatim}

\begin{verbatim}
## [1] 6908
\end{verbatim}

\begin{verbatim}
## [1] 7
\end{verbatim}

\begin{verbatim}
## [1] 6908
\end{verbatim}
\end{frame}

\section{Analyse des Correspondances
Factorielles}\label{analyse-des-correspondances-factorielles}

\begin{frame}{Analyse des Correspondances Factorielles}
\pandocbounded{\includegraphics[keepaspectratio]{Presentation_files/figure-beamer/unnamed-chunk-3-1.pdf}}
\end{frame}

\begin{frame}{Présentation générale de l'ACF}
\phantomsection\label{pruxe9sentation-guxe9nuxe9rale-de-lacf}
\end{frame}

\begin{frame}{Description des variables}
\phantomsection\label{description-des-variables}
\begin{block}{Variables actives:}
\phantomsection\label{variables-actives}
Étiquette\_dpe, Étiquette\_ges, Année\_construction, Nombre\_appartement
\end{block}

\begin{block}{Variables quantitatives supplémentaires:}
\phantomsection\label{variables-quantitatives-suppluxe9mentaires}
Consommation.annuelle\ldots, Surface\_habitable,
Émission\_ges\_5\_usages, Coût\_total\_5\_usages
\end{block}

\begin{block}{variable exclue:}
\phantomsection\label{variable-exclue}
Adresse, Numero\_dpe, Conso\_5\_usages\_ef,
Consommation.annuelle.moyenne\ldots{}
\end{block}
\end{frame}

\begin{frame}{Qualité globale de l'ACF}
\phantomsection\label{qualituxe9-globale-de-lacf}
\begin{columns}[T]
\begin{column}{0.5\linewidth}
\pandocbounded{\includegraphics[keepaspectratio]{Presentation_files/figure-beamer/unnamed-chunk-4-1.pdf}}
\end{column}

\begin{column}{0.5\linewidth}
\end{column}
\end{columns}
\end{frame}

\begin{frame}{interprétation de l'axe 1}
\phantomsection\label{interpruxe9tation-de-laxe-1}
\end{frame}

\begin{frame}{interprétation de l'axe 2}
\phantomsection\label{interpruxe9tation-de-laxe-2}
\end{frame}

\begin{frame}{Analyse des individus}
\phantomsection\label{analyse-des-individus}
\end{frame}

\begin{frame}{synthèse finale}
\phantomsection\label{synthuxe8se-finale}
\end{frame}

\section{ACM}\label{acm}

\begin{frame}[fragile]{ACM}
\begin{verbatim}
## 'data.frame':    7156 obs. of  11 variables:
##  $ Conso_totale              : num  25.3 17.8 138.2 221.1 364.1 ...
##  $ Conso_Moy_log             : num  1.81 1.62 2.19 3.51 1.29 ...
##  $ Conso_com                 : num  6.32 6.32 2.63 2.63 2.63 ...
##  $ etiquette_dpe             : chr  "C" "C" "F" "E" ...
##  $ etiquette_ges             : chr  "C" "C" "B" "B" ...
##  $ annee_construction        : int  2021 2021 1980 1945 1974 1945 1910 1967 1947 1973 ...
##  $ surface_habitable_logement: num  49.1 81.8 32 30.2 27.1 22.7 142 50 83.3 75.1 ...
##  $ conso_5.usages_ef         : num  3405 6672 4944 4094 5145 ...
##  $ emission_ges_5_usages     : num  713 1431 360 302 915 ...
##  $ cout_total_5_usages       : num  429 782 892 934 494 ...
##  $ nombre_appartement        : int  11 11 65 71 278 12 33 21 12 83 ...
##  - attr(*, "na.action")= 'omit' Named int [1:779] 28 33 36 40 67 89 96 104 105 107 ...
##   ..- attr(*, "names")= chr [1:779] "28" "33" "36" "40" ...
\end{verbatim}
\end{frame}

\begin{frame}[fragile]{Conversion des variables qualitatives en factor
et decoupage en classe}
\phantomsection\label{conversion-des-variables-qualitatives-en-factor-et-decoupage-en-classe}
\begin{verbatim}
## 'data.frame':    7156 obs. of  11 variables:
##  $ Conso_totale              : Factor w/ 4 levels "[4.43,28.1]",..: 1 1 4 4 4 3 4 1 4 4 ...
##  $ Conso_Moy_log             : Factor w/ 4 levels "[0.225,1.37]",..: 2 2 3 4 1 4 4 1 4 4 ...
##  $ Conso_com                 : Factor w/ 3 levels "[2.09,2.63]",..: 3 3 1 1 1 1 1 1 1 1 ...
##  $ etiquette_dpe             : Factor w/ 7 levels "A","B","C","D",..: 3 3 6 5 4 4 4 4 5 4 ...
##  $ etiquette_ges             : Factor w/ 7 levels "A","B","C","D",..: 3 3 2 2 4 2 4 4 5 2 ...
##  $ annee_construction        : Factor w/ 4 levels "[1.61e+03,1.94e+03]",..: 4 4 3 1 3 1 1 2 2 3 ...
##  $ surface_habitable_logement: Factor w/ 4 levels "[4.8,38.8]","(38.8,57]",..: 2 4 1 1 1 1 4 2 4 4 ...
##  $ conso_5.usages_ef         : Factor w/ 4 levels "[405,5.57e+03]",..: 1 2 1 1 1 1 4 3 4 2 ...
##  $ emission_ges_5_usages     : Factor w/ 4 levels "[18.7,764]","(764,1.54e+03]",..: 1 2 1 1 2 1 4 3 4 1 ...
##  $ cout_total_5_usages       : Factor w/ 4 levels "[112,679]","(679,946]",..: 1 2 2 2 1 1 4 2 4 4 ...
##  $ nombre_appartement        : Factor w/ 4 levels "[10,20]","(20,33]",..: 1 1 4 4 4 1 2 2 1 4 ...
##  - attr(*, "na.action")= 'omit' Named int [1:779] 28 33 36 40 67 89 96 104 105 107 ...
##   ..- attr(*, "names")= chr [1:779] "28" "33" "36" "40" ...
\end{verbatim}
\end{frame}

\begin{frame}[fragile]{Mise en Oeuvre de l'ACM}
\phantomsection\label{mise-en-oeuvre-de-lacm}
\begin{block}{Inertie}
\phantomsection\label{inertie-1}
\begin{verbatim}
##        eigenvalue percentage of variance cumulative percentage of variance
## dim 1  0.32898428             11.3879172                          11.38792
## dim 2  0.25400705              8.7925517                          20.18047
## dim 3  0.18733237              6.4845820                          26.66505
## dim 4  0.17066563              5.9076564                          32.57271
## dim 5  0.15028373              5.2021291                          37.77484
## dim 6  0.14259164              4.9358646                          42.71070
## dim 7  0.12868983              4.4546479                          47.16535
## dim 8  0.11962922              4.1410114                          51.30636
## dim 9  0.11540260              3.9947055                          55.30107
## dim 10 0.11332981              3.9229549                          59.22402
## dim 11 0.11258706              3.8972443                          63.12127
## dim 12 0.10833263              3.7499755                          66.87124
## dim 13 0.10552963              3.6529487                          70.52419
## dim 14 0.10163376              3.5180916                          74.04228
## dim 15 0.09870000              3.4165384                          77.45882
## dim 16 0.09454890              3.2728465                          80.73167
## dim 17 0.09214541              3.1896488                          83.92131
## dim 18 0.08508121              2.9451187                          86.86643
## dim 19 0.07542610              2.6109036                          89.47734
## dim 20 0.07402703              2.5624742                          92.03981
## dim 21 0.06315375              2.1860914                          94.22590
## dim 22 0.04861602              1.6828623                          95.90876
## dim 23 0.03936316              1.3625709                          97.27134
## dim 24 0.03700662              1.2809983                          98.55233
## dim 25 0.02879911              0.9968921                          99.54923
## dim 26 0.01302235              0.4507737                         100.00000
\end{verbatim}
\end{block}

\begin{block}{Graphique des Inerties}
\phantomsection\label{graphique-des-inerties}
\pandocbounded{\includegraphics[keepaspectratio]{Presentation_files/figure-beamer/unnamed-chunk-10-1.pdf}}
\end{block}
\end{frame}

\section{}\label{section}

\section{Mise en Oeuvre de la clasification non supervisée à partir de
l'ACM}\label{mise-en-oeuvre-de-la-clasification-non-supervisuxe9e-uxe0-partir-de-lacm}

\begin{frame}{Mise en Oeuvre de la clasification non supervisée à partir
de l'ACM}
On commence par faire par faire une classification hierarchique
ascendante sur la sortie de l'ACM avec consolidation puis faire la
méthodes \textbf{k-means}
\end{frame}

\begin{frame}{reste}
\phantomsection\label{reste}
\end{frame}

\end{document}
